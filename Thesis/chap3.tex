\chapter{Preliminaries}
\label{chap:Preliminaries}

In this chapter, we review the preliminary concepts.

\section{Linear Temporal Logics}
\label{sec:pre}

Let $\AP$ be a set of {\em atomic propositions} and $\alphabet = 2^{\AP}$ be the 
{\em alphabet}. We call each element of $\alphabet$ an {\em state}. A {\em 
trace} is a sequence $ \state_0\state_1\cdots$, where  $\state_i\in \alphabet$ 
for every $i\geq 0$. The set of all finite (resp., infinite) traces over 
$\alphabet$ is denoted by $\alphabet^*$ (resp., $\alphabet^\omega$). Throughout 
the thesis, we denote finite traces by the letter $\fintrace$, and infinite 
traces by the  letter $\inftrace$. We denote the empty trace by $\emptyword$. 
Finally, For a finite trace $\fintrace = \state_0\state_1\cdots \state_n$, by 
$\fintrace^i$ we mean trace suffix $\state_i\state_{i+1}\cdots\state_n$ of 
$\fintrace$.

\subparagraph{LTL Syntax.} Formulas in {\em linear temporal logic} 
(\LTL)~\cite{mp79} are defined using the following grammar:
$$\varphi ::=  p \; \mid \; \neg\varphi \; \mid \; \varphi\vee\varphi \; \mid 
\; \X\varphi \; \mid \; \varphi \, \U \, \varphi$$
where $p   \in \AP$ is an atomic proposition, $\U$ is the ``until'' operator, 
and $\X$ is the ``next operator''. Additionally, we allow the following 
operators as syntactic sugar, each of which is defined in terms of the above 
ones: $ \tru = p \vee  \neg p, \fals = \neg \tru, \varphi_1 \wedge \varphi_2  = 
\neg (\neg \varphi_1 \vee \neg \varphi_2), \textbf{F} \varphi = \tru 
\, \textbf{U} \, \varphi $,and $\textbf{G} \varphi  = \neg \F (\neg \varphi)$. 
Formulas without temporal operators are called state formulas.

\subparagraph{LTL Semantics.}  The semantics of \LTL is defined w.r.t. infinite 
traces. Let $\inftrace = s_0s_1\cdots$ be an infinite trace in 
$\alphabet^\omega$, $i \geq 0$, and $\models$ denote the {\em satisfaction} 
relation. The semantics of 
\LTL is defined as follows:

\begin{tabbing}
\hspace{5mm} \=  $\inftrace, i \models p$ \hspace*{2cm} \= iff \hspace*{.5cm} 
\= $p \in \state_i$ \\
\> $\inftrace, i \models \neg \varphi$ \> iff \> $\inftrace,i \not \models 
\varphi$ \\
\> $\inftrace, i \models \varphi_1\vee\varphi_2$ \> iff \> $\inftrace, i 
\models 
\varphi_1 \; \text{ or } \; \inftrace, i \models \varphi_2$ \\
\> $\inftrace, i \models \X \varphi$ \> iff \> $\inftrace, i+1 \models \varphi$ 
\\
\> $\inftrace, i \models \varphi_1 \, \U \, \varphi_2$  \> iff \> $\exists k 
\geq 
i : \inftrace, k \models \varphi_2 \; \text{ and } \; \forall j \in [i, k): 
\inftrace, j \models \varphi_1$.
\end{tabbing}
Also, $\inftrace \models \varphi$ holds \; iff \; $\inftrace, 0 
\models \varphi$ holds. For example, consider the following 
\emph{request/acknowledgment} \LTL formula:
$$\varphi_{ra} = \G\Big(\neg a \wedge \neg r\Big) \; \vee \, \Big((\neg a \, \U 
\, r) \, \wedge \, \F a\Big)$$
This formula requires that (1) if a request is emitted (i.e., $r=\tru$), 
then it should eventually be acknowledged (i.e., $a=\tru$), and (2) an 
acknowledgment happens only in response to a request.

\section{LTL for Runtime Verification}

\subsection{Finite LTL (FLTL)}

The semantics of \LTL is defined over infinite traces. In the context of 
runtime verification, since a system only generates finite traces, the standard 
\LTL semantics does not seem to be the appropriate formalism. Finite \LTL 
(denoted \FLTL~\cite{mp95}) allows us to reason about finite traces for 
verifying properties at run time. The syntax of \FLTL is identical to that of 
\LTL and the semantics is based on the truth values 
$\mathbb{B}_2=\{\top,\bot\}$. The semantics of \FLTL for atomic propositions and 
Boolean operators are identical to those of \LTL. We now recall the semantics of 
\FLTL for the temporal operators. Let $\varphi$, $\varphi_1$, and $\varphi_2$ be 
\LTL formulas, $\fintrace = \state_0\state_1 \cdots \state_{n}$ be a non-empty 
finite trace, and $\models_F$ denote satisfaction in~\FLTL. We have

\[
\left[\fintrace \models_F \X\, \varphi\right]  = 
\begin{cases}
	[\fintrace^1 \models_F \varphi] & \text{if}\; 
\fintrace^1 \neq \emptyword \\
	\bot & \text{\normalfont otherwise}
\end{cases}
\]
and
\[
\left[\fintrace \models_F \varphi_1 \,\U\, \varphi_2\right] 
= 
\begin{cases}
\top & \text{if}\; \exists k \in [0,n]: ([\fintrace^k 
\models_F \varphi_2] = \top) \;\wedge \; (\forall \ell \in [0,k), 
[\fintrace^\ell \models_F \varphi_1] = \top)\\
	\bot & \text{\normalfont otherwise}
\end{cases}
\]

%\begin{definition} [\FLTL semantics]
%Let $\varphi$, $\varphi_1$, and $\varphi_2$ be \LTL formulas, and $\fintrace = 
%\state_0\state_1 \cdots \state_{n}$ be a finite trace.
%
%\begin{align*}
%\left[\fintrace \models_{\text{\normalfont F}} \X\, \varphi\right] &= 
%\begin{cases}
%	[\fintrace^1 \models_{\text{\normalfont F}} \varphi] & \text{if }~~~~ 
%\fintrace^1 \neq \emptyword \\
%	\bot & \text{\normalfont otherwise}
%\end{cases}\\ \\
%\left[\fintrace \models_{\text{\normalfont F}} \varphi_1 \,\U\, \varphi_2\right] 
%&= 
%\begin{cases}
%\top & \text{if }~~~~\exists k \in [0,n]: [\fintrace^k 
%\models_{\text{\normalfont F}} \varphi_2] = \top \;\;\wedge \\
%	&~~~~~~~\forall l \in [0,k) : [\fintrace^l \models_{\text{\normalfont 
%F}} \varphi_1] = 
%\top\\
%	\bot & \text{\normalfont otherwise}
%\end{cases}
%\end{align*}
%\qed
%\end{definition}





To illustrate the difference between \LTL and \FLTL, let $\varphi = \F p$ and  
$\fintrace=\state_0\state_1\cdots \state_{n}$. If $p \in \state_i$ for some $i 
\in [0, n]$, then we have $[\fintrace \models_F \varphi] = \top$. Otherwise, 
$[\fintrace \models_F \varphi] = \bot$, and this holds even if the program under 
inspection extends $\fintrace$ in the future to a state where $p$ becomes true. 

% \begin{definition} [\FLTL Monitor]
% \label{def:monitorf}
% Let $\varphi$ be an \LTL formula. The \FLTL~{\em monitor} of $\varphi$ is 
% the unique deterministic finite state machine $\monitor^{\varphi}_F = 
% (\alphabet, Q, q_0, \delta, \lambda)$, where $Q$ is a set of states, 
% $q_0$ is the initial state, $\delta \subseteq Q \times \alphabet \times Q$ is 
% the transition relation, and $\lambda: Q \rightarrow \mathbb{B}_2$ is a 
% function such that:
% 
%  \begin{equation*}
% \lambda(\delta(q_0, u)) = 
% \begin{cases} \left[u \models_F \varphi\right] & \text{if } 
% u \neq \epsilon \\
% 
% \tru & \text{if } u = \epsilon
% \end{cases}
%  \end{equation*}
% for every finite word $u \in \alphabet^*$.
%  \qed
%  \end{definition}
% \noindent In~\cite{mp95}, the authors propose an algorithm that takes as 
% input an \LTL formula and constructs as output an \FLTL monitor. 

\subsection{3-valued LTL}
\label{sec:ltl3}

As illustrated above, for a finite trace $\fintrace$, \FLTL ignores the 
possible future extensions of $\fintrace$, when evaluating a formula. The 
3-valued semantics of \LTL (denoted \LTLtri~\cite{bls11}) evaluates \LTL 
formulas for finite traces with an eye on possible future extensions. In 
\LTLtri, the set of truth values is $\mathbb{B}_3=\{\top,\bot, ?\}$, where 
`$\top$' (resp., `$\bot$') denotes that the formula is permanently satisfied 
(resp., violated), no matter how the 
current execution extends, and `?' denotes an unknown \truthvalue; i.e., there exist 
an extension that can falsify the formula, and another extension that can 
truthify the formula.
%
% \begin{definition}[\LTLtri semantics]
% \label{def:ltl3}

Now, let $\fintrace \in \alphabet^{*}$ be a non-empty finite trace. The truth 
value 
of an \LTLtri formula $\varphi$ with respect to $\fintrace$, denoted by 
$[\fintrace \models_3 \varphi]$, is defined as follows:
 \begin{equation*}
 	\left[\fintrace \models_3 \varphi \right] = 
 		\begin{cases} \top & \text{if }~~~~~\forall \inftrace \in 
\alphabet^\omega : \fintrace\inftrace \models \varphi\\
 		\bot & \text{if }~~~~~\forall \inftrace \in \alphabet^\omega : 
\fintrace\inftrace \not \models \varphi\\
 		? & \text{otherwise}.
 	\end{cases}
 \end{equation*}
% \qed
% \end{definition} 


The \LTLtri~{\em monitor} of a formula $\varphi$ is the unique deterministic 
finite state machine $\monitor^{\varphi}_3 = \{ \alphabet, Q, q_0, \delta, 
\lambda \}$, where $Q$ is a set of states, $q_0$ is the initial state, $\delta: Q 
\times \alphabet \to Q$ is the transition function, and $\lambda: Q 
\rightarrow \mathbb{B}_3$, is a function such that:
\[
\lambda(\delta(q_0, \fintrace)) = \left[\fintrace\models_3 \varphi\right]
\]
for every finite trace $\fintrace  \in \alphabet^*$.
%\qed
% \end{definition}
%
In~\cite{bls11}, the authors introduce an algorithm that takes as input an \LTL 
formula and constructs as output an \LTLtri monitor. For example, 
Fig. 3.1 shows the \LTLtri monitor for the \LTL formula $\varphi = a \, \U \, b$.




 \begin{figure}[H]
     \centering
  \begin{tikzpicture}[->,>=stealth',shorten >=1pt,auto,node distance=2.8cm,
semithick, scale=0.8, every node/.style={scale=0.8}, initial text={},
font=\normalsize]
   \tikzstyle{state}=[circle,thick,draw=black,fill=none,minimum size=10mm]
\node[state, accepting] (A)                    {$q_\bot$};
  \node[initial,state]         (B) [above right of=A, xshift= -6mm] {$q_0$};
  \node[state, accepting]         (C) [below right of=B, xshift= -6mm]
{$q_\top$};

 \path  (B) edge [loop above] node  {$a \wedge \neg b$} (B)
        (B) edge node [left, yshift=2mm] {$\neg a \wedge \neg b$}
(A)
        (B) edge node [right, yshift=2mm] {$b$} (C)
        (A) edge [loop below] node  {$\mathit{true}$} (A)
        (C) edge [loop below] node  {$\mathit{true}$} (C);
\end{tikzpicture} 

   \caption{\LTLtri monitor for $\varphi = a \, \U \, b$.}
    \end{figure}
   \label{fig:LTL3}
%\todo{Place this in a figure environment.}

%\section{Decentralized Synchronous Monitoring }
%\label{sec:synch}



\subsection{RV-LTL}


\RVLTL \cite{bls10}, which we will denote in this thesis by \LTLfour, refines 
the truth value ? into $\bot_p$ and~$\top_p$. That is, 
$\mathbb{B}_4=\{\top,\TT,\FT,\bot\}$. More specifically, evaluation of 
a formula in \LTLfour agrees with \LTLtri if the verdict is $\bot$ or 
$\top$. Otherwise, (i.e., when the verdict in \LTLtri is ?), \LTLfour 
utilizes \FLTL to compute a more refined truth value.

Now, let $\fintrace \in \alphabet^{*}$ be a finite trace. The truth value of an 
\LTLfour formula $\varphi$ with respect to $\fintrace$, denoted by $[\fintrace 
\models_4 \varphi]$, is defined as follows:
\begin{equation*}
 	\left[\fintrace \models_{4} \varphi \right] = 
 		\begin{cases} \top & \text{if }~~~~~[\fintrace \models_3 
\varphi] = \top\\
 		\bot & \text{if }~~~~~[\fintrace \models_3 \varphi] = 
\bot\\
 		\TT & \text{if }~~~~~[\fintrace \models_3 \varphi] =? \; \wedge 
\; [\fintrace \models_F \varphi] = \top\\ 

 		\FT & \text{if }~~~~~[\fintrace \models_3 \varphi] =? \; \wedge 
\; [\fintrace \models_F \varphi] = \bot\\ 
 	\end{cases}
 \end{equation*}



\begin{figure}[H]
\centering
\begin{tikzpicture}[->,>=stealth',shorten >=1pt,auto,node distance=2.8cm,
                    semithick, initial text={}, initial where=right]
  \tikzstyle{every state}=[scale=0.65, every node/.style={scale=0.65}]

  \node[initial,state] (A)                {$\top_p$};
  \node[state]         (B) [left of  = A] {$\bot_p$};
  \node[state, accepting]         (D) [below of = B] {$\top$};
  \node[state, accepting]         (C) [below of = A] {$\bot$};

  \path (A) edge              node {$\neg a \wedge r$} (B)
            edge              node {$a \wedge \neg r$} (C)
            edge [loop above] node {$\neg a \wedge \neg r$} (A)
            edge              node {$a\wedge r$} (D)
        (B) edge [loop above] node {$\neg a \wedge r$} (B)
            edge              node {$a$} (D)
        (C) edge [loop below] node {$\tru$} (D)
        (D) edge [loop below] node {$\tru$} (D);


\end{tikzpicture}     \caption{\LTLfour monitor of $\varphi_{ra}$.}
\end{figure}
  \label{fig:RVltl}





 
 
The \LTLfour~{\em monitor} of a formula $\varphi$ is the unique deterministic 
finite state machine $\monitor^{\varphi}_4 = \{ \alphabet, Q, q_0, \delta, 
\lambda\}$, where $Q$ is a set of states, $q_0$ is the initial state, $\delta: Q 
\times \alphabet \to Q$ is the transition function, and $\lambda: Q 
\rightarrow \mathbb{B}_4$, is a function 
such that:
\[
\lambda(\delta(q_0, \fintrace)) = \left[\fintrace\models_4 \varphi\right]
\]
for every finite trace $\fintrace  \in \alphabet^*$.
%\qed
% \end{definition}
%
In \cite{bls11} , the authors introduce an algorithm that takes as input an \LTL 
formula and constructs as output an \LTLfour monitor. For example, 
Fig. 3.2 shows the \LTLfour monitor for the 
request/acknowledgement formula 
$\varphi_{ra} = \G(\neg a \wedge \neg r) \; \vee \, [(\neg a \, \U \, r) 
\,\wedge \, \F a]$.








