\chapter{Conclusion}
\label{chap:Conclusion}

\section{Summary}


In this thesis, we studied synchronous and asynchronous runtime verification of 
distributed systems and presented distributed monitoring algorithms 
for this purpose, which allow three-valued LTL monitoring. In particular,

\begin{itemize}

\item we proposed a synchronous monitoring algorithm that copes with $f$ crash 
failures in a distributed setting. The algorithm solves the synchronous 
monitoring problem in $f+1$ rounds of communication, where at each round each 
local monitor broadcasts a message, receives messages from other monitors, and 
performs local computation based on the received messages and computes a message 
to be sent in the subsequent round. We proposed an automata-based algorithm 
where each local monitor's message is a set of monitor states that are reachable 
from the current monitor state (according to the input automaton) by the set of 
possible global states from viewpoint of that local monitor. Therefore our 
algorithm reduces the message size overhead from $|\AP|$ to $\log(m_\monstate)$ 
where $m_\monstate$ is the number of outgoing transitions from the current 
monitor state $\monstate$. We showed that the input automaton (that is employed 
in each monitor's algorithm) must satisfy the following condition:
$$\forall \monstate \in Q.~ |\intersection| = 1$$ where $Q$ is the set of all 
monitor states in the input automaton, and $\intersection$ denotes the 
intersection of all verdict sets emitted by local monitors. Therefore we 
introduced an algorithm to construct an \Exltl~that satisfies the aforementioned 
condition.

\item We proposed an algorithm for distributed crash-resilient asynchronous RV that consistently monitors the system under inspection without any communication between monitors. Each local monitor emits a verdict set solely based on its own partial observation, and the intersection of the verdict sets will be the same as the verdict computed by a centralized monitor that has full view of the system. 

\end{itemize}


\section{Future Work}

Some open problems for further research are as follows:

\begin{itemize}

\item In our framework the fault model was crash failure, i.e., the monitors can only fail by crashing. From a more practical perspective, it would be interesting to address more severe, e.g., Byzantine failures.

\item In the asynchronous monitoring, although we assumed the monitors are 
asynchronous wait-free processes, however, it was supposed that the global state 
of the system changes synchronously, i.e., all monitors observe the same 
global state. We can relax the timing model so that monitors observe, 
communicate, and emit verdicts between any two global states. 

\item Our results in the decentralized asynchronous monitoring can 
theoretically be transformed to more practical refinements such as message 
passing frameworks.

\item It would of course be interesting to extend our results to the case where 
the input to the monitors is a sequence of global states and each monitor 
produces a sequence of verdict sets, one per each global state. 

\end{itemize}
