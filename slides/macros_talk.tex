

\newcommand{\FLTL}{\textrm{FLTL}\xspace}
\newcommand{\LTL}{\textrm{LTL}\xspace}

\newcommand{\AP}{\mathrm{AP}}  %
\newcommand{\cL}{\mathcal{L}}
\newcommand{\lattice}{\cL}

\newcommand{\logic}{\mathfrak{L}}
\newcommand{\SemL}[2]{{[#1 \models #2]_{\logic}}}
\newcommand{\Sem}[2]{{[#1 \models #2]}}
\newcommand{\SemF}[2]{{[#1 \models #2]_F}}
\newcommand{\SemFF}[2]{{[#1 \models #2]_4}}
\newcommand{\SemT}[2]{{[#1 \models #2]_3}}
\newcommand{\SemRV}[2]{{[#1 \models #2]_{RV}}}
\newcommand{\SemFf}[3]{{[#1,#2 \models #3]_{F}}}
\newcommand{\SemRVf}[3]{{[#1,#2 \models #3]_{RV}}}
\newcommand{\SemTf}[3]{{[#1,#2 \models #3]_3}}
\newcommand{\Semf}[3]{{[(#1,#2) \models #3]}}
%
\newcommand{\SemW}[2]{{[#1 \models #2]_{-}}}
\newcommand{\SemS}[2]{{[#1 \models #2]_{+}}}
\newcommand{\SemSW}[2]{{[#1 \models #2]_{\pm}}}
\newcommand{\SemWS}[2]{{[#1 \models #2]_{\mp}}}
\newcommand{\SemLTL}[2]{{[#1 \models #2]_{\omega}}}

\newcommand{\semExp}[2]{{\mathtt{ftl4}(#1,#2)}}
\newcommand{\fltlRew}[2]{{\mathtt{fltlRew}(#1,#2)}}


\newcommand{\Def}[1]{\alert{#1}}

\newcommand{\cA}{{\mathcal{A}}}
\newcommand{\cF}{{\mathcal{F}}}

\newcommand{\cM}{{\mathcal{M}}}
\newcommand{\BA}{\mathrm{BA}}
\newcommand{\NFA}{\mathrm{NFA}}
\newcommand{\DFA}{\mathrm{DFA}}
\renewcommand{\phi}{\varphi}

\newcommand{\true}{\ensuremath{\mathit{true}}\xspace} 
\newcommand{\false}{\ensuremath{\mathit{false}}\xspace}
\newcommand{\X}{\mathit{X}}
\newcommand{\nX}{\bar{\X}}    %
\newcommand{\U}{\mathrel{\mathit{U}}} %
\newcommand{\R}{\mathrel{\mathit{R}}} %

\newcommand{\sem}[1]{\ensuremath[\![#1]\!]}
\newcommand{\junit}{jUnit\xspace}
\newcommand{\junitrv}{jUnit$^{\mathrm{RV}}$\xspace}
\newcommand{\ltl}{\textrm{LTL}\xspace}

\newcommand{\WX}{\ensuremath{\operatorname{\overline{X}}}\xspace}
\newcommand{\G}{\ensuremath{\operatorname{G}}\xspace}
\newcommand{\F}{\ensuremath{\operatorname{F}}\xspace}
\renewcommand{\S}{\ensuremath{\operatorname{S}}\xspace}

\newcommand{\Until}{\mathit{U}}
\newcommand{\Release}{\mathit{R}}
\renewcommand{\complement}[1]{{\overline{#1}}}

\newcommand{\pbot}{{\bot^p}}
\newcommand{\ptop}{{\top^p}}
\newcommand{\meet}{\sqcap}
\newcommand{\join}{\sqcup}

\newcommand{\dual}[1]{\overline{#1}}


\renewcommand{\phi}{\varphi}
\newcommand{\nphi}{{\neg\phi}}

\newcommand{\monitor}{\ensuremath{{\sf monitor}}\xspace}
\newcommand{\step}{\ensuremath{{\sf step}}\xspace}
\newcommand{\decide}{\ensuremath{{\sf decide}}\xspace}

\newcommand{\emptiness}{\ensuremath{\text{\sf non-empty}}\xspace}

\newcommand{\extrapolate}{\ensuremath{{\sf extrapolate}}\xspace}
\newcommand{\abstractfct}{\ensuremath{\alpha}}

\lstdefinelanguage{forLTL}
{morekeywords={
  let, in, true, false,
  always, historically, alwaysinpast,
  between, betweeninpast,
  eventually, once, eventuallyinpast,
  from, after, frominpast,
  holding, holdinginpast,
  never, neverinpast,
  next, previous, nextinpast,
  nextn, previousn, nextninpast,
  occurring, occurringinpast,
  releases, triggered, releasesinpast,
  until, since, untilinpast,
  upto, before, uptoinpast,
  required, req, optional, opt, weak,
  inclusive, incl, exclusive, excl,
  and, or, implies, equals, not,
  if, then, else,
  rejecton, accepton,
  assert, declare, define,
  allof, someof, noneof, exactlyoneof,
  enumerate, list, as, in,
  with, without,
  timed,
  alw, even  %
},
sensitive=true,
morecomment=[l]{--},
morestring=[b]"
}

\lstset{frame=none, 
        basicstyle=\ttfamily, 
        keywordstyle=\bfseries, 
        mathescape=true,
        commentstyle=\color[gray]{0.5}, 
        breaklines=true, 
        breakatwhitespace=true,
        showstringspaces=false,
        xleftmargin=3ex
}
